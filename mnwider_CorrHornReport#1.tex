\documentclass[11pt]{article}

% --- Page and typography ---
\usepackage[margin=1in]{geometry}
\usepackage{setspace}
\setstretch{1.08}
\usepackage{parskip}
\usepackage{microtype}
\usepackage[T1]{fontenc}
\usepackage{lmodern}

% --- Math & units ---
\usepackage{amsmath,amssymb}
\usepackage{siunitx}
\sisetup{detect-all, per-mode=symbol, exponent-product=\cdot, output-decimal-marker=.}

% --- Tables & graphics ---
\usepackage{booktabs}
\usepackage{tabularx}
\usepackage{graphicx}
\usepackage{subcaption}
\usepackage{float}
\usepackage{caption}
\captionsetup{font=small, labelfont=bf, justification=centering, singlelinecheck=true}
\graphicspath{{./}{./figs/}{./images/}}
\renewcommand{\arraystretch}{1.15}

% --- Lists & color ---
\usepackage{xcolor}
\usepackage{enumitem}
\setlist{leftmargin=1.4em, itemsep=0.25em}

% --- Links ---
\usepackage{hyperref}
\hypersetup{colorlinks=true, linkcolor=black, urlcolor=blue, citecolor=black}

% Make \autoref output bold hyperlinks (affects figures/tables/equations)
\makeatletter
\let\oldautoref\autoref
\renewcommand{\autoref}[1]{\textbf{\oldautoref{#1}}}
\makeatother
% Handy bold, hyperlinked Figure reference macro for plain "Figure N" mentions
\newcommand{\figref}[1]{\textbf{\hyperref[#1]{Figure~\ref*{#1}}}}

% --- Headers / footers & section styling (formatting only) ---
\usepackage{fancyhdr, lastpage}
\pagestyle{fancy}
\fancyhf{}
\lhead{\footnotesize Corrugated Horn Design Report}
\rhead{\footnotesize Maximus Nwider}
\cfoot{\footnotesize \thepage\ of \pageref{LastPage}}
\usepackage{titlesec}
\titleformat{\section}{\large\bfseries}{\thesection}{0.5em}{}
\titleformat{\subsection}{\normalsize\bfseries}{\thesubsection}{0.5em}{}
\titleformat{\subsubsection}{\normalsize\bfseries\itshape}{\thesubsubsection}{0.5em}{}

% --- Convenience placeholders (kept, optional to use) ---
\newcommand{\ph}[1]{\textcolor{gray}{\textit{[Placeholder: #1]}}}

% --- Title (unchanged) ---
\title{\textbf{Corrugated Horn Design Report}}
\author{Maximus Nwider}
\date{}
\begin{document}
\maketitle

\tableofcontents
\hrule\vspace{0.8em}

% =====================================================


% =====================================================
\section{Physical Realization}
\subsection{Extracted Geometric Parameters Provided From Screenshots}
\begin{table}[H]
  \centering
  \caption{Extracted Flare Geometric Parameters}
  \label{tab:geom}
  \begin{tabularx}{\linewidth}{@{}l l X@{}}
    \toprule
    \textbf{Variable (symbol)} & \textbf{Value} & \textbf{Brief description} \\
    \midrule
    Length $L$ & \SI{80.0}{\milli\meter} & Distance from throat plane to aperture plane. \\
    Number of corrugations $N$ & 6 & Corrugation count. \\
    Period $p$ & \SI{13.33}{\milli\meter} & Center-to-center spacing of corrugations. \\
    Groove width $w_g$ & \SI{8.8}{\milli\meter} & Axial width of each corrugation groove. \\
    Groove depth $d_k$ & $\{2,4,6,8,10,10\}\,\si{mm}$ & Radial depth of the $k$-th groove. \\
    Wall thickness $t_{\text{wall}}$ & \SIrange{3}{4}{\milli\meter} & Metal wall thickness of the horn body. \\
    Tooth $t_{\text{tooth}}$ & \SI{4.53}{\milli\meter} & Metal between each groove; $t_{\text{tooth}}=p-w_g$. \\
    Flare half-angle $\alpha$ & \SI{6.77}{\degree} . \\
    \bottomrule
  \end{tabularx}
\end{table}
\begin{figure}[H]
  \centering
  \begin{subfigure}[b]{0.4\textwidth}
    \centering
    \includegraphics[width=0.85\linewidth]{FLARE.png}
    \caption{Matlab rendered cross-section.}
    \label{fig:matlab_xsec}
  \end{subfigure}
  \begin{subfigure}[b]{0.4\textwidth}
    \centering
    \includegraphics[width=0.75\linewidth]{image.png}
    \caption{AutoCAD rendered cross-section.}
    \label{fig:acad_xsec}
  \end{subfigure}
  \caption{To-scale physical realization of the flare section.}
\end{figure}


% =====================================================
\section{Simulation in HFSS}
\subsection{Simulation Semantics and Setup}
\begin{itemize}
  \item \textbf{Waveport}: To initiate a simulation in HFSS, a waveport must be imposed on the 3D object, to define the source and direction of input radiation, i.e a source. The waveport was assigned at the throat of the design.
  \item \textbf{PEC cap}: Additionally, to Run the HFSS 3D simulation, a small cylinder of material type:”PEC”
(Perfect Electrical Conductor) is placed on the opposing face of the previously defined waveport,
dimensions of this object do not matter for our purposes, as long as it covers the waveport, does
not dimensionally overlap, and is not too thick, ideally make it as thin as possible.
  \item \textbf{Open Region}: An arbitrary simulation Boundary is placed, around the 3D design, representing a region of free space to define the simulation boundary around the design, to tell the software where to stop Finite Element analysis.
\end{itemize}
\begin{figure}[H]
  \centering
  \begin{subfigure}[b]{0.48\textwidth}
    \centering
    % \includegraphics[width=0.75\linewidth]{"Screenshot 2025-09-08 at 12.49.09 PM.png"}
    \includegraphics[width=0.62\linewidth]{Screenshot 2025-09-08 at 12.49.09 PM.png}
    \caption{Enclosing open-region.}
    \label{fig:open_region}
  \end{subfigure}\hfill
  \begin{subfigure}[b]{0.48\textwidth}
    \centering
    % \includegraphics[width=0.62\linewidth]{"Screenshot 2025-09-08 at 1.15.32 PM.png"}
    \includegraphics[width=0.5\linewidth]{Screenshot 2025-09-08 at 1.15.32 PM.png}
    \caption{3D model with waveport (red).}
    \label{fig:waveport}
  \end{subfigure}
  \caption{HFSS domain and excitation.}
\end{figure}

\subsection{Frequency Sweep and Post-Processing}
\textbf{Solution Setup}, A linear sweep from 5GHz to 9 GHz of "fast type" is run with a maximum delta of $0.05$. Despite ChatGPT's Suggested frequency band of 7-8GHz, you will see that the return losses, at that high of a frequency are much higher, and the "ideal low loss" return loss occurs at \textbf{6.75GHz}, as opposed to 7.5GHz suggested earlier.
\subsection{Far-Field Results and HPBW}
\begin{figure}[H]
  \centering
  \includegraphics[width=0.85\linewidth]{farfield.png}
  \caption{3D far-field plot showing a maximum of 10.4dB Overall Gain in the primary lobe (at \SI{6.75}{GHz}, swept over $\theta$ and $\phi$).}
  \label{fig:farfield3d}
\end{figure}

\begin{figure}[H]
  \centering
  \includegraphics[width=0.98\linewidth]{radplot.png}
  \caption{Radiation patterns swept used to estimate average HPBW.}
  \label{fig:radplot}
\end{figure}

\begin{figure}[H]
  \centering

  % Top row: two smaller panels
  \begin{subfigure}[b]{0.48\textwidth}
    \centering
    \includegraphics[width=\linewidth]{ph45.png}
    \caption{$\phi=\SI{45}{\degree}$}
    \label{fig:phi45}
  \end{subfigure}\hfill
  \begin{subfigure}[b]{0.48\textwidth}
    \centering
    \includegraphics[width=\linewidth]{ph90.png}
    \caption{$\phi=\SI{90}{\degree}$}
    \label{fig:phi90}
  \end{subfigure}

  \vspace{0.6em}

  % Bottom row: large panel
  \begin{subfigure}[b]{0.98\textwidth}
    \centering
    \includegraphics[width=0.95\linewidth]{phi0.png}
    \caption{$\phi=\SI{0}{\degree}$}
    \label{fig:phi0}
  \end{subfigure}

  \caption{Principal–plane cuts used to extract the half–power beamwidth (HPBW). Markers indicate the \SI{-3}{dB} points bounding the primary lobe in each plane.}
  \label{fig:hpbw_planes}
\end{figure}

\textbullet\ As seen in \autoref{fig:hpbw_planes} (\textbf{\subref{fig:phi45}},\,\textbf{\subref{fig:phi90}},\,\textbf{\subref{fig:phi0}}), the separation in $\theta$ between the \SI{-3}{dB} markers lies in the \SIrange{52}{54}{\degree} interval, consistent with the computed average. An independent CST run (Abdelkareem) produced a similar HPBW within $\pm\SI{2}{\degree}$. Accordingly, the HPBW for this geometry is assessed as \SIrange{51}{55}{\degree}.




\subsection{Return Loss \texorpdfstring{$S_{11}$}{S11}}
\begin{figure}[H]
  \centering
  \includegraphics[width=1\linewidth]{return.png}
  \caption{Return loss over frequency (\SI{5}{GHz}--\SI{9}{GHz}).}
  \label{fig:return}
\end{figure}

\textbullet\ As Shown by \figref{fig:return}, for input signals with frequencies hovering around the $6.75GHz$ mark, an extremely minimal portion of the original signal actually gets reflected, acting as a near perfect match, with very minimal loss, this in-fact is what should be represented as the center frequency, for reference when plotting the Far-Field plot, rather than the previously suggested $7.5GHz$ Value.

\textbullet\ Additionally, moving forward with $6.75GHz$ as the locally optimal center frequency, and taking the bandwidth as $\pm 10dB$ from the center frequency, the frequency band would fall between $5.8GHz$ to $7.3GHz$, as opposed to the previously suggested bandwidth of $7-8GHz$.


% Requires: \usepackage{subcaption} \usepackage{float} \usepackage{siunitx}

% Requires: \usepackage{subcaption} \usepackage{float} \usepackage{siunitx}

\subsection{Polarization Performance}
% Consistent full-row sizing for all panels
\newlength{\rowwidth}\setlength{\rowwidth}{0.98\textwidth}
\newlength{\rowheight}\setlength{\rowheight}{0.25\textheight} % adjust if you want taller/shorter

\subsubsection{Cross-Polarization}
\begin{figure}[H]
  \centering

  \begin{subfigure}[b]{\rowwidth}
    \centering
    \includegraphics[width=\linewidth,height=\rowheight,keepaspectratio]{Cross_phi0.png}
    \caption{$\phi=\SI{0}{\degree}$}
    \label{fig:cross-phi0}
  \end{subfigure}


  \begin{subfigure}[b]{\rowwidth}
    \centering
    \includegraphics[width=\linewidth,height=\rowheight,keepaspectratio]{Cross_phi45.png}
    \caption{$\phi=\SI{45}{\degree}$}
    \label{fig:cross-phi45}
  \end{subfigure}


  \begin{subfigure}[b]{\rowwidth}
    \centering
    \includegraphics[width=\linewidth,height=\rowheight,keepaspectratio]{Cross_phi90.png}
    \caption{$\phi=\SI{90}{\degree}$}
    \label{fig:cross-phi90}
  \end{subfigure}

  \caption{Cross-polarization cuts at $\phi=\SI{0}{\degree}$, $\SI{45}{\degree}$, and $\SI{90}{\degree}$. Plots are scaled for readability (dynamic range $\pm\SI{10}{dB}$) at \SI{6.75}{GHz}.}
  \label{fig:cross-cuts}
\end{figure}

\subsubsection{Co-Polarization}
\begin{figure}[H]
  \centering

  \begin{subfigure}[b]{\rowwidth}
    \centering
    \includegraphics[width=\linewidth,height=\rowheight,keepaspectratio]{Co_phi0.png}
    \caption{$\phi=\SI{0}{\degree}$}
    \label{fig:co-phi0}
  \end{subfigure}

  \medskip

  \begin{subfigure}[b]{\rowwidth}
    \centering
    \includegraphics[width=\linewidth,height=\rowheight,keepaspectratio]{Co_phi45.png}
    \caption{$\phi=\SI{45}{\degree}$}
    \label{fig:co-phi45}
  \end{subfigure}

  \medskip

  \begin{subfigure}[b]{\rowwidth}
    \centering
    \includegraphics[width=\linewidth,height=\rowheight,keepaspectratio]{Co_phi90.png}
    \caption{$\phi=\SI{90}{\degree}$}
    \label{fig:co-phi90}
  \end{subfigure}

  \caption{Co-polarization cuts at $\phi=\SI{0}{\degree}$, $\SI{45}{\degree}$, and $\SI{90}{\degree}$. Plots are scaled for readability (dynamic range $\pm\SI{10}{dB}$) at \SI{6.75}{GHz}.}
  \label{fig:co-cuts}
\end{figure}

% =====================================================
\section{Suggested Design -> pending}
\subsection{Topology and Required Parameters}
\begin{figure}[H]
  \centering
  \includegraphics[width=0.98\linewidth]{suggested.png}
  \caption{Suggested topology for improved performance.}
  \label{fig:suggested_topo}
\end{figure}
\begin{figure}[H]
  \centering
  \includegraphics[width=0.6\linewidth]{pattern.png}
  \caption{Target 3D far-field pattern.}
  \label{fig:target_pattern}
\end{figure}
\begin{figure}[h!]
    \centering
    \includegraphics[width=0.4\linewidth]{2d_suggested.png}
    \caption{2D Cross-Section}
    \label{fig:placeholder}
\end{figure}

% =====================================================
\section{Discussion and Next Steps}

\end{document}
