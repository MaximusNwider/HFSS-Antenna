
\documentclass[11pt]{article}
\usepackage[margin=1in]{geometry}
\usepackage{amsmath,amssymb}
\usepackage{siunitx}
\usepackage{booktabs}
\usepackage{hyperref}
\usepackage{xcolor}
\usepackage{enumitem}
\usepackage{graphicx}

\hypersetup{colorlinks=true, linkcolor=black, urlcolor=blue}
\sisetup{detect-all, per-mode=symbol, exponent-product=\cdot, output-decimal-marker=.}

\begin{document}

\begin{center}
{\LARGE \textbf{Corrugated Horn Design Report}\\[6pt]
\end{center}



\section*{Dimensional Data}


\begin{itemize}[leftmargin=1.4em]
  \item Length (Throat to Aperture): $L=\SI{80.0}{mm}$.


  \item Wall thickness: $t_{\text{wall}}\in\SIrange{3}{4}{mm}$ (noted range).
  \item Number of corrugations: $N=6$.
  \item Groove centers from throat plane (mm):\\
        $\{\,\SI{6.7}{},\ \SI{20.0}{},\ \SI{33.3}{},\ \SI{46.6}{},\ \SI{59.9}{},\ \SI{73.2}{}\,\}$.
  \item Groove radial depths (mm): $\{\,\SI{2}{},\ \SI{4}{},\ \SI{6}{},\ \SI{8}{},\ \SI{10}{},\ \SI{10}{}\,\}$.
  \item Groove axial width: $w_g=\SI{8.8}{mm}$.
  \item Groove pitch (center-to-center): $p=\SI{13.33}{mm}$.
\end{itemize}

\begin{itemize}[leftmargin=1.4em]
  \item \textbf{$2a$}: the full aperture diameter ($D_a=\SI{54}{mm}$).
  \item \textbf{$D_t$}: the throat diameter ($\SI{35}{mm}$).
  \item \textbf{$p$}: corrugation period (center-to-center), $\SI{13.33}{mm}$.
  \item \textbf{$w_g$}: groove axial width, $\SI{8.8}{mm}$.
  \item \textbf{$d$}: groove depth; sequence $\{2,4,6,8,10,10\}\,\si{mm}$.
  \item \textbf{$t_{\text{wall}}$}: metal wall thickness, \SIrange{3}{4}{mm}.
  \item \textbf{$t_{\text{tooth}}$}: axial land between grooves, $\SI{4.53}{mm}$.
  \item \textbf{$\alpha$}: flare half-angle, $\approx \SI{6.77}{\degree}$.

\end{itemize}



\begin{itemize}[leftmargin=1.4em]

  \item \textbf{Aperture diameter} \textbf{$2a$}: equals $D_a=\SI{54}{mm}$ (i.e., $a=R_a=\SI{27}{mm}$).
  \item \textbf{Throat diameter} $D_t$: equals \SI{35}{mm}.
  \item \textbf{Flare half-angle} $\alpha$: defined by the linear inner-wall cone between radii $R_t$ and $R_a$ over $L$.
  \item \textbf{Groove period} $p$: corrugation center-to-center spacing.
  \item \textbf{Groove width} $w_g$: axial width of each rectangular corrugation slot.
  \item \textbf{Tooth (land) thickness} $t_{\text{tooth}}$: axial metal between adjacent slots, $t_{\text{tooth}}=p-w_g$.
  \item \textbf{Groove depth} $d_k$: radial reduction of the inner wall within the $k$-th groove.
\end{itemize}

\section*{4.\quad Geometric Model (inner/outer radii)}
Let $z\in[0,L]$ measure distance from the throat plane toward the aperture.
The \emph{baseline} (ungrooved) inner radius is the linear cone
\begin{equation}
r_{\text{cone}}(z)=R_t + (R_a - R_t)\,\frac{z}{L}.
\end{equation}
With rectangular corrugations, the actual inner radius is
\begin{equation}
r_{\text{in}}(z)=r_{\text{cone}}(z) - \sum_{k=1}^{N} d_k\,\chi_k(z),\qquad
\chi_k(z)=\begin{cases}
1, & z\in[z_{ck}-\tfrac{w_g}{2},\,z_{ck}+\tfrac{w_g}{2}],\\[2pt]
0, & \text{otherwise},
\end{cases}
\end{equation}
where $z_{ck}$ is the $k$-th groove center. The outer wall is smooth at approximately constant thickness; using the screenshot range,
\begin{equation}
r_{\text{out}}(z)=r_{\text{cone}}(z)+t_{\text{wall}},\qquad t_{\text{wall}}\in[\SI{3}{mm},\SI{4}{mm}].
\end{equation}

\paragraph{Flare half-angle.} From the data
\begin{equation}
\alpha=\arctan\!\left(\frac{R_a-R_t}{L}\right)
= \arctan\!\left(\frac{\SI{27}{mm}-\SI{17.5}{mm}}{\SI{80}{mm}}\right)
\approx \SI{6.77}{\degree}.
\end{equation}

\paragraph{Tooth (land) thickness.} With $p=\SI{13.33}{mm}$ and $w_g=\SI{8.8}{mm}$,
\begin{equation}
t_{\text{tooth}}=p-w_g=\SI{13.33}{mm}-\SI{8.8}{mm}=\SI{4.53}{mm}.
\end{equation}

\section*{5.\quad Corrugation Rules from the Screenshots}
The screenshots encode the standard subwavelength constraints:
\begin{align}
p &\approx \frac{\lambda}{3},\label{eq:p}\\
w_g &\approx 0.22\,\lambda,\label{eq:wg}\\
d_{\max} &\approx \frac{\lambda}{4}, \label{eq:dmax}
\end{align}
with depths increasing along $z$ and saturating near $\lambda/4$ (the last two grooves are \SI{10}{mm} deep).

\section*{6.\quad Operating Wavelength and Frequency (cross-check)}
Using \eqref{eq:p}--\eqref{eq:dmax} with the provided numbers:

\paragraph{From the pitch:}
\begin{equation*}
\lambda_{(p)} \approx 3p = 3\times \SI{13.33}{mm} = \SI{39.99}{mm}.
\end{equation*}

\paragraph{From the groove width:}
\begin{equation*}
\lambda_{(w_g)} \approx \frac{w_g}{0.22} = \frac{\SI{8.8}{mm}}{0.22} = \SI{40.0}{mm}.
\end{equation*}

\paragraph{From the max depth:}
\begin{equation*}
\lambda_{(d)} \approx 4\,d_{\max} = 4\times \SI{10.0}{mm} = \SI{40.0}{mm}.
\end{equation*}

All three estimates are consistent at $\lambda \approx \SI{40.0}{mm}$. Using the speed of light $c=\SI{299792458}{m/s}$,
\begin{equation*}
f=\frac{c}{\lambda}=\frac{\SI{299792458}{m/s}}{\SI{0.0400}{m}} \approx \SI{7.495}{GHz}.
\end{equation*}
Hence the design center is approximately \textbf{\SI{7.5}{GHz}}, which aligns with a \SIrange{7}{8}{GHz} operating band suggested by the geometry.

\section*{What the Legends Mean (for drawings)}
\begin{itemize}[leftmargin=1.4em]
  \item \textbf{$2a$}: the full aperture diameter ($D_a=\SI{54}{mm}$).
  \item \textbf{$D_t$}: the throat diameter ($\SI{35}{mm}$).
  \item \textbf{$p$}: corrugation period (center-to-center), $\SI{13.33}{mm}$.
  \item \textbf{$w_g$}: groove axial width, $\SI{8.8}{mm}$.
  \item \textbf{$d$}: groove depth; sequence $\{2,4,6,8,10,10\}\,\si{mm}$.
  \item \textbf{$t_{\text{wall}}$}: metal wall thickness, \SIrange{3}{4}{mm}.
  \item \textbf{$t_{\text{tooth}}$}: axial land between grooves, $\SI{4.53}{mm}$.
  \item \textbf{$\alpha$}: flare half-angle, $\approx \SI{6.77}{\degree}$.
  \item \textbf{A/B/C}: drawing segments (upstream/smooth/corrugated). Corrugations begin at $z=\ell_s=\SI{2.3}{mm}$.
\end{itemize}

\section*{8.\quad Notes confined to the screenshots}
\begin{itemize}[leftmargin=1.4em]
  \item All numerical values above are \emph{exactly} those shown or algebraic combinations thereof.
  \item No assumptions beyond the screenshot text and numbers have been introduced.
  \item Bandwidth, return loss, and pattern performance are not stated in the screenshots and thus are not asserted here.
\end{itemize}

\bigskip
\noindent\textit{End of report.}

\end{document}
"""
with open("/mnt/data/corrugated_horn_report.tex", "w") as f:
    f.write(content)
print("Wrote LaTeX file to /mnt/data/corrugated_horn_report.tex")
